\chapter{第三章~~基于流形正则的上下文感知相关滤波跟踪}
%\addcontentsline{toc}{chapter}{第三章~~基于静态空间学习的相关滤波跟踪} %目录里的章节名
\label{chap02}

\section{动机}
尽管相关滤波类跟踪已经被研究多年,但是,其固有的样本循环假设也引入了严重的冗余,这样不利于学到一个有效的分类器。



\section{做法}
在本节中,我们开发了一个快速的流形正则上下文感知相关滤波跟踪算法,来挖掘不同类型样本间局部的流形结构信息。

首先,不同于只利用一个基样本的传统相关滤波类跟踪,我们使用了一系列基样本周围的上下文样本,并对其进行了流形结构假设。然后,考虑到这些样本中存在的流形结构,我们在相关滤波(CF)学习中,引入了一个线性图拉普拉斯正则项。幸运的是,这个优化问题能够利用快速傅里叶变换得到一个闭式解,因而能够高效地实现。
\section{数据集上的结果}

大量在OTB100和VOT2016数据集上的实验评估表明,提出的跟踪算法在准确性和鲁棒性方面,表现优于几个最先进的算法的。特别地,我们的跟踪器能够在单个CPU上以28fps的速度实时运行。



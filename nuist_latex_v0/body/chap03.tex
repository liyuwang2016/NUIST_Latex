\chapter{第四章~~通过颜色聚类和时空正则相关学习的互补型跟踪}
\label{chap03}


\section{动机}
最近,一种简单而高效名叫Staple的跟踪算法在效率和准确度方面,取得了很有潜力的结果。它配备有判别式相关滤波(DCFs)跟踪器和颜色直方图的互补学习器,这种互补特性使得它对颜色变化和形变都具有很强的适应性。然而,它也有一些缺点:(1)Staple对于所有序列,只使用标准颜色直方图,量化标准相同,并没有考虑目标在每个序列中的具体结构信息,从而影响其区分目标和背景的能力。(ii)Staple中使用的标准DCFs是有效的,但是存在多余的边界效应问题,导致一些具有挑战性的场景中的跟踪失败。

\section{做法}
为了解决这些问题,我们提出了基于颜色聚类和时空正则相关回归的互补跟踪器(CSCT)。 所提出的CSCT包含两个互补的部分,可以自适应地处理每个序列的显着颜色变化和形变问题:首先,我们设计了一个新颖的基于颜色聚类的直方图模型,该模型首先自适应地将第一帧目标的颜色划分为几个颜色聚类中心,构建自适应颜色直方图,这样一来模型就能适应严重的目标形变问题。另外,我们提出学习时空正则化的CFs,既能缓解边界效应,在目标剧烈变化时,又能提供一个比原Staple里标准DCF更稳健的表观模型。
\section{数据集上的结果}

与Staple相比,我们的CSCT仅仅利用简单的手工特征,在OTB100、Temple-Color和VOT2016 这三个数据集上,分别提升了5.9\%,3.4\%和1.5\%。此外,我们的CSCT甚至超过了几个最先进的复杂深度网络跟踪器。






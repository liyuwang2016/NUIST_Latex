%%%%%%%%%%%%%%%%%%%%%%%%%%%%%%%%%%%%%%%%%%%%%%%%%%%%%%%%%%%%%%%%%%%%%% %
%
% 南京信息工程大学硕士论文 XeLaTeX 模版 —— 主文件 main.tex
% 版本:2018.10
% 制作时参考了各个大学的模板,作者:自动化学院 樊佳庆 (E-mail: fjq199407@163.com)
% 使用环境:Win7 + CTeX v2.9.2.164_Full + WinEdit
% CTex下载地址:https://mirrors.tuna.tsinghua.edu.cn/ctex/legacy/2.9/
% 注意,安装1.3G完整版的
%%%%%%%%%%%%%%%%%%%%%%%%%%%%%%%%%%%%%%%%%%%%%%%%%%%%%%%%%%%%%%%%%%%%%%

\documentclass[12pt, a4paper, openany, twoside]{book}

% 配置字体
% 英文字体设置特别推荐方案(Windows,需要安装 Adobe 字体),现代
\usepackage{fontspec}
\usepackage{xltxtra,xunicode}
\usepackage[CJKnumber,CJKchecksingle,BoldFont]{xeCJK}
\usepackage{amsmath}
\usepackage{amssymb}
% \usepackage{mathspec}
\setmainfont[Mapping=tex-text]{Times New Roman}
\setsansfont[Mapping=tex-text]{Arial}
\setmonofont{Consolas}
% \setmathfont{Times New Roman}

% 中文字体设置,使用的是 Adobe 字体,保证了在 Adobe Reader / Acrobat 下优秀的显示效果
\setCJKmainfont[BoldFont={Adobe Heiti Std},ItalicFont={Adobe Kaiti Std}]{Adobe Song Std}
\setCJKsansfont{Adobe Heiti Std}
\setCJKmonofont{Adobe Fangsong Std}

% 定义字体名称,可在此添加自定义的字体
\setCJKfamilyfont{song}{Adobe Song Std}
\setCJKfamilyfont{hei}{Adobe Heiti Std}
\setCJKfamilyfont{kai}{Adobe Kaiti Std}
\setCJKfamilyfont{fs}{Adobe Fangsong Std}
\setCJKfamilyfont{xkai}{STXingkai}

% 自动调整中英文之间的空白
% \punctstyle{quanjiao}
\XeTeXlinebreaklocale "zh"      %中文断行
\XeTeXlinebreakskip = 0pt plus 1pt %1pt左右弹性间距
% 其他字体宏包


% 宏包配置文件
\input{setup/packages}

% 格式文件
\input{setup/format}

\begin{document}

% 定义所有的图片文件在 figures 子目录下
\graphicspath{{figures/}}

% 前言
\frontmatter
\pagenumbering{Roman}


%\cdegree{\textbf{快速目标跟踪算法的有效正则化研究}}
\ctitle{\textbf{基于时空上下文学习的快速目标跟踪算法研究}}
\etitle{Fast Object Tracking via Spatio-temporal context Learning}



% 编辑姓名等个人信息
\theauthor{{\quad\;}樊~佳~庆}
\teacher{{\quad\;}张开华~~教授}
\themajor{{\quad\;}系~统~科~学}
\direction{{\quad\;}视频单目标跟踪}
\college{{\quad\;}自动化学院}

%%中文摘要的具体内容
\cabstract{
  尽管已研究多年,快速并且高性能的目标跟踪算法依然缺失。其中一个关键问题是,视频帧中的时空上下文信息并未被充分利用,而这部分信息对于相关滤波跟踪的性能影响重大。为了解决上述问题,本文主要基于时空上下文学习对快速目标跟踪算法进行研究,使之能更有效地利用时空上下文信息建模,主要贡献概括为如下2点:

  \begin{asparaenum}
  \item 尽管相关滤波类跟踪已经被研究多年,但是,其固有的样本循环假设也引入了严重的冗余,这样不利于学到一个有效的分类器。在本文中,我们开发了一个快速的流形正则上下文感知相关滤波跟踪算法,来挖掘不同类型样本间局部的流形结构信息。首先,不同于只利用一个基样本的传统相关滤波类跟踪,我们使用了一系列基样本周围的上下文样本,并对其进行了流形结构假设。然后,考虑到这些样本中存在的流形结构,我们在相关滤波(CF)学习中,引入了一个线性图拉普拉斯正则项。幸运的是,这个优化问题能够利用快速傅里叶变换得到一个闭式解,因而能够高效地实现。大量在OTB100和VOT2016数据集上的实验评估表明,提出的跟踪算法在准确性和鲁棒性方面,表现优于几个最先进的算法的。特别地,我们的跟踪器能够在单个CPU上以28fps的速度实时运行。
  \item 最近,一种简单而高效名叫Staple的跟踪算法在效率和准确度方面,取得了很有潜力的结果。它配备有判别式相关滤波(DCFs)跟踪器和颜色直方图的互补学习器,这种互补特性使得它对颜色变化和形变都具有很强的适应性。然而,它也有一些缺点:(1)Staple对于所有序列,只使用标准颜色直方图,量化标准相同,并没有考虑目标在每个序列中的具体结构信息,从而影响其区分目标和背景的能力。(ii)Staple中使用的标准DCFs是有效的,但是存在多余的边界效应问题,导致一些具有挑战性的场景中的跟踪失败。为了解决这些问题,我们提出了基于颜色聚类和时空正则相关回归的互补跟踪器(CSCT)。 所提出的CSCT包含两个互补的部分,可以自适应地处理每个序列的显着颜色变化和形变问题:首先,我们设计了一个新颖的基于颜色聚类的直方图模型,该模型首先自适应地将第一帧目标的颜色划分为几个颜色聚类中心,构建自适应颜色直方图,这样一来模型就能适应严重的目标形变问题。另外,我们提出学习时空正则化的CFs,既能缓解边界效应,在目标剧烈变化时,又能提供一个比原Staple里标准DCF更稳健的表观模型。与Staple相比,我们的CSCT仅仅利用简单的手工特征,在OTB100、Temple-Color和VOT2016 这三个数据集上,分别提升了5.9\%,3.4\%和1.5\%。此外,我们的CSCT甚至超过了几个最先进的复杂深度网络跟踪器。

%  \item 在目标跟踪领域,动态时空信息几乎没有被人提及。比如,最近的空间正则相关滤波跟踪(SRDCF)类的跟踪算法虽然创新地提出了对滤波器的空间约束方法,但是固定了对滤波器的约束矩阵,使其不能适应目标与相关滤波器的快速变化,最终滤波器会不断退化,致使跟踪失败。为了解决此问题,本文提出了动态时空正则学习模型,在总目标函数中引入了动态时空正则项来同时学出惩罚矩阵和相关滤波器,使得相关滤波器被更好地约束,能缓解遮挡等造成的过拟合问题。
  \end{asparaenum}
}

\ckeywords{目标跟踪;相关滤波;时空正则;上下文学习}


%%英文摘要的具体内容
\eabstract{
Despite years of research, fast and high-performance target tracking algorithms are still missing.
%
One of the key issues is that the spatio-temporal context information in frame video is not fully utilized, and this part of information has a significant impact on correlation filtering tracking performance.
%
In order to solve the above problems, in this thesis, we develop fast object tracking algorithms mainly based on spatio-temporal context learning.
The main contribution of this study is summarized as follows:

(1)%
Despite the demonstrated success of numerous correlation filter (CF) based tracking approaches, their assumption of circulant structure of samples introduces significant redundancy to learn an effective classifier.
%
In this paper, we develop a fast manifold regularized context-aware correlation tracking algorithm that mines the local manifold structure information of different types of samples.
%
First, different from the traditional CF based tracking that only uses one base sample, we employ a set of contextual samples near to the base sample, and impose a manifold structure assumption on them. Afterwards, to take into account the manifold structure among these samples, we introduce a linear graph Laplacian regularized term into the objective of CF learning. Fortunately, the optimization can be efficiently solved in a closed form with fast Fourier transforms (FFTs), which contributes to a highly efficient implementation. Extensive evaluations on the OTB100 and VOT2016 datasets demonstrate that the proposed tracker performs favorably against several state-of-the-art algorithms in terms of accuracy and robustness. Especially, our tracker is able to run in real-time with 28 fps on a single CPU.

(2)Recently, a simple, yet effective and efficient tracker named Staple has achieved promising performance in terms of efficiency and accuracy on a series of visual tracking benchmarks.
%
Staple is equipped with complementary learners of discriminative correlation filters (DCFs) and color histograms, which are robust to both color changes and deformations.
%
However, it has some drawbacks:
%
(i) Staple only employs standard color histograms with the same quantization step for all sequences, which does not consider the specific structural information of target in each sequence, thereby affecting its discriminative capability to separate target from background.
%
(ii) The standard DCFs are efficient but suffer from unwanted boundary effects, leading to failures in some challenging scenarios.
%
To address these issues, we present a dual color clustering and spatio-temporal regularized correlation regressions based complementary tracker (CSCT).
%
The proposed CSCT includes two components with complementary merits to adaptively deal with significant color variations and deformations for each sequence:
%
First, we design a novel color clustering based histogram model that first adaptively divides the colors of the target in the 1st frame into several cluster centers, and then the cluster centers are taken as references to construct adaptive color histograms for targets in the coming frames, which enable to adapt significant target deformations.
%
Second, we propose to learn spatio-temporal regularized CFs, which not only enable to avoid boundary effects but also provides a more robust appearance model than the discriminative CFs in Staple in the case of large appearance variations.
%
Compared with Staple, our CSCT with handcrafted features achieves a gain of 5.9\%, 3.4\% and 1.5\% on OTB100, Temple-Color and VOT2016 benchmarks in terms of AUC and EAO scores, respectively. Moreover, our CSCT performs favorably against several state-of-the-art trackers including the deep learning based trackers.


}

\ekeywords{Visual tracking; correlation filter; spatio-temporal regularization; context learning}
\makecover
  % 输入封面
\originality           % 创建独创性说明和授权说明



%%生成目录
\defaultmenufont  % 设置默认目录字体和行间距
\tableofcontents  % 正式生成目录
%\cleardoublepage
% 插图目录,我们不用
% \listoffigures
% 表格目录,我们也不用
% \listoftables
%\cleardoublepage

%% 创建摘要放在目录后面
\makeabstract
\defaultfont
\mainmatter
% 正文章节
\defaultfont
\renewcommand{\thefootnote}{\arabic{footnote}} % 页脚
\chapter*{\hfill 第一章~~绪论 \hfill} %正文里的第一章标题,这里的*表示不标章节序号
\addcontentsline{toc}{chapter}{第一章~~绪论}  %目录里的第一章标题
\label{chap00}

\section{研究背景}
视觉跟踪是目前国内外研究的热点之一,在计算机视觉中众多的应用,如
视频监控,运动分析,自动驾驶,举几个例子~\cite{li2013survey,wangvisual,ali2016visual,zhang2018visual,zhang2018visualtracker}。 尽管近年来取得了很大进展,但开发一种鲁棒跟踪算法仍然是一个挑战,主要由于目标外观的显著变化而引起的例如光照变化,快速运动,姿势变化,部分遮挡和背景混乱等。正因为遭受了这些挑战,一个鲁棒的表示在视觉跟踪中更显得尤为重要,因此在过去的几十年中它引起了广泛的关注。





\section{国内外研究现状}


\subsection{快速相关滤波类目标跟踪}

相关滤波类,在线训练,速度快,效果较好。

\subsection{神经网络类目标跟踪}

孪生网络类,训练慢,测试快,效果好。


\section{文章简介与结构}

\subsection{本文工作简介}

基于时空上下文学习,我们做了2个工作。

(1)提出了基于流形正则的上下文感知相关滤波跟踪。

(2)提出了通过颜色聚类和时空正则相关学习的互补型跟踪。



\subsection{本文结构}

二、三章分别详细介绍这两个工作。



\chapter{第二章~~相关工作和评价指标} %正文里的章节名
%\addcontentsline{toc}{chapter}{第二章~~相关工作和评价指标} %目录里的章节名

\label{chap01}

\section{相关工作}

\subsection{基于相关滤波的目标跟踪}
最近,判别式相关滤波跟踪(DCFs)在视觉跟踪方面引起了广泛的关注,由于它们在效率和稳健性方面的优势。使用DCFs 进行视觉跟踪从MOSSE~\cite{bolme2010visual}开始,它在频率域中用几个样本学习CFs,又利用快速傅里叶变换(FFTs)高效地计算,最终能以669 帧/每秒(FPS)运行。在~\cite{henriques2012exploiting}中,Henriques 等人首先探索了循环密集样品的结构与核化嵌入,学习出CFs 进行快速跟踪。在~\cite{henriques2015high}中,Henriques等人进一步改进了~\cite{henriques2012exploiting}中的CF 跟踪器,把特征表示从原始图像强度升级到梯度方向直方图。Ma等人~\cite{ma2015hierarchical}利用不同层次深层特征间的互补,使用由粗至细的特征搜索策略,学习更有效的视觉CFs 跟踪,显著提高了相关滤波在OTB100 数据集上的性能。最近,Danelljan 等人在空间上提出了一系列的基于正则化CF 的跟踪器~\cite{danelljan2015learning,danelljan2016adaptive,danelljan2016beyond},取得了令人印象深刻的性能。空间正则化DCF (SRDCF) 跟踪器~\cite{danelljan2015learning}试图抑制学习CFs的边界效应,它利用了具有高斯形状的空间正则化权值。基于~\cite{danelljan2015learning}, ~\cite{danelljan2016adaptive}提出了一种自适应去污方案,因而学习到了更有效的CFs,它能自适应地学习可靠性并消除各训练样本中被污染的样本。在~\cite{danelljan2016beyond}中,是在各种特征映射的连续空间域内学习CFs,能够获得亚像素级别的跟踪精度。

\subsection{基于流形正则的目标跟踪}

流形正则化通常用于半监督的有标记和无标记样本学习~\cite{belkin2006manifold, chang2017semisupervised, yu2013harry},它构造了一个Laplacian 图来利用样本利用特征空间的隐式几何结构。例如,在特征空间分析中,Chang和Yang ~\cite{chang2017semisupervised}利用标记和未标记的训练数据进行更多的研究可靠的特征空间选择算法。此外,在视觉跟踪里,Yu~\cite{yu2013harry}等人利用具有时空约束的流形结构进行跟踪,在现实世界的人员定位和监控场景中具有较好的效果。Bai 和Tang~\cite{bai2012robust} 用了一个在线拉普拉斯正则化的排序支持向量机,来估计视觉跟踪的对象位置。为了更好地使用无标记数据和流形结构样本空间,Hu等人~\cite{hu2017manifold}提出了一个基于流形正则化DCF的跟踪器,增加了循环位移的样本并利用一个块优化策略,可以有效地通过FFTs 计算。Zhuang 等人~\cite{zhuang2014visual}为视觉跟踪构建了一个判别式的稀疏相似图,它是基于多任务的拉普拉斯正则反稀疏表示。

\subsection{整合多类估计的跟踪}
有一种常用的用来减少不精确预测的策略,就是将一系列方法的估计结合起来,这样跟踪器的缺点就会得到相应的补偿。
%
在~\cite{kwon2010visual, kwon2011tracking}里, Kwon等人利用互补的基础跟踪器,结合不同的观测模型和运动模型,然后在一个采样的框架中集成他们的估计。
%
在\cite{wang2014ensemble}中, Wang 和 Yeung 通过一个因子的HMM结合了几个独立的跟踪器,同时建模了目标轨迹和每个跟踪器随着时间变化的可靠性。
%
不同于使用不同类型的跟踪器,多专家最小熵跟踪器保留了一个过去模型的集合,并且根据熵判断准则选择出了一个最优的的预测。
%
在~\cite{bertinetto2016staple}中, Bertinetto等人直接合并了两个常见的岭回归得分,这里的局部表示利用HOG特征,全局表示利用了颜色特征,最后两者一同互补地工作。
%






\section{评价指标}
\subsection{OTB数据集}
OTB100数据集包含了100个测试视频序列,使用了两个典型的评价标准。第一个是精度,即定位误差小于阈值的帧所占的百分比。定位误差定义为跟踪中心与真实边界框中心之间的距离(以像素为单位)。另一个是成功率,即重叠率大于阈值的帧所占的百分比。
%
重叠得分被定义为
\begin{equation}
OS = \frac{{\left| {{G_{rec}} \cap {T_{rec}}} \right|}}{{\left| {{G_{rec}} \cup {T_{rec}}} \right|}},
\end{equation}
这里的${{G_{rec}}}$ 表示真实边界框,${{T_{rec}}}$ 是跟踪到的边界框. 跟踪结果的得分只要高于阈值$t$ 就会被认为是一次成功的跟踪。
%
最终的成功率图显示了重叠分数大于$t$的帧所占百分比,其中$t$在0到1 之间变化。利用曲线下面积(AUC)对评估后的跟踪器进行排序~\cite{wu2015object}。
%
\subsection{VOT数据集}
VOT数据集每年都以挑战赛的形式更新,一般每年的VOT数据集都包含了60 个视频序列,它使用了预期的平均覆盖率(EAO)作为评价标准,这个得分是根据准确性,稳健性这两个指标计算出来的。准确性是指预测到的边界框和真实边界框之间的平均覆盖率。而稳健性度量了在跟踪中跟丢目标的次数,反映了跟踪算法针对不同视频的稳定性~\cite{kristan2015visual}。

\subsection{Temple-Color数据集}
Temple-Color数据集包含了128个颜色视频序列,旨在于评价跟踪算法在彩色视频序列上的性能~\cite{liang2015encoding}。

\chapter{第三章~~基于流形正则的上下文感知相关滤波跟踪}
%\addcontentsline{toc}{chapter}{第三章~~基于静态空间学习的相关滤波跟踪} %目录里的章节名
\label{chap02}

\section{动机}
尽管相关滤波类跟踪已经被研究多年,但是,其固有的样本循环假设也引入了严重的冗余,这样不利于学到一个有效的分类器。



\section{做法}
在本节中,我们开发了一个快速的流形正则上下文感知相关滤波跟踪算法,来挖掘不同类型样本间局部的流形结构信息。

首先,不同于只利用一个基样本的传统相关滤波类跟踪,我们使用了一系列基样本周围的上下文样本,并对其进行了流形结构假设。然后,考虑到这些样本中存在的流形结构,我们在相关滤波(CF)学习中,引入了一个线性图拉普拉斯正则项。幸运的是,这个优化问题能够利用快速傅里叶变换得到一个闭式解,因而能够高效地实现。
\section{数据集上的结果}

大量在OTB100和VOT2016数据集上的实验评估表明,提出的跟踪算法在准确性和鲁棒性方面,表现优于几个最先进的算法的。特别地,我们的跟踪器能够在单个CPU上以28fps的速度实时运行。



\include{body/chap03}
%\include{body/chap04}  %如果第5章用不到的话,你就直接删了它。

%% 结论
\chapter*{\hfill 结  论 \hfill}
\addcontentsline{toc}{chapter}{结  论}

本文从时空上下文学习的角度,对快速相关滤波目标跟踪算法进行了研究。%

(1)提出了基于流形正则的上下文感知相关滤波跟踪。


(2)提出了通过颜色聚类和时空正则相关学习的互补型跟踪。

这些提出的跟踪算法有效地利用了时空上下文信息,实验证明确实提升了跟踪算法的性能。

%\cleardoublepage
 %\backmatter


% 参考文献
\defaultfont
\wuhao
\bibliographystyle{zjugbno}
\bibliography{body/reference}
\addcontentsline{toc}{chapter}{参考文献}
%\cleardoublepage  %我认为这里不用偶数页空白


%% 附录A是辅助性的公式推导或图表,不是必须的
%\defaultfont
%\begin{appendix}
%  \input{appendix/chapA}
%\end{appendix}
%\cleardoublepage

\defaultfont
%%发表的文章列表,是必须要的
\include{appendix/publications}
%\cleardoublepage

%% 致谢是必须要的
\chapter*{\hfill 致  谢 \hfill}
\addcontentsline{toc}{chapter}{致  谢}

感谢宋慧慧老师,张开华老师,刘青山老师及在研究过程中帮助过我的朋友们。感谢南京信息工程大学自动化学院提供的研究平台。

特别致谢:孙师兄。

%\cleardoublepage %我认为这里也不用偶数页空白

% 授权书,已删,与独创性合并
%\chapter*{}
%\addcontentsline{toc}{chapter}{南京信息工程大学学位论文版权使用授权书}
%\renewcommand{\baselinestretch}{1.61}
%\vspace{-0.48cm}
%\authorization
%\cleardoublepage

\end{document}
